\documentclass[compress, aspectratio=1610]{zzu-slide}

\addbibresource{ref.bib}

\hypersetup{
	colorlinks=true,
	linkcolor=black,
	filecolor=green,
	urlcolor=red,
	citecolor=blue,
}

\usepackage{background}
\backgroundsetup{
    scale=2, % 水印大小
    color=red, % 水印颜色
    opacity=0.1, % 水印透明度
    angle=45, % 水印角度
    position=current page.center,
    % vshift=2cm,
    % hshift=2cm,
    contents={Non-commercial use only.} % 水印内容
}
\setbeamertemplate{background}{\BgMaterial}

% 取消注释可自定义颜色
% 深蓝色示例
% \definecolor{xmublue}{RGB}{0,68,170}
% \definecolor{xmulightblue}{RGB}{200,222,255}
% 橙色示例
% \definecolor{xmublue}{RGB}{249,138,0}
% \definecolor{xmulightblue}{RGB}{255,231,200}
% 粉色示例
% \definecolor{xmublue}{RGB}{219,115,188}
% \definecolor{xmulightblue}{RGB}{255,206,239}

% 取消注释可自定义字体(例如使用 macOS 下更好看的字体)
% \setsansfont{SF Pro}
% \setCJKsansfont{PingFang SC}

% 浅色主题 / 深色主题(注释即为浅色)
% \dark

% 底部控制放映的导航条(取消注释则不显示)
\nav

% 封面的毕设题目
\title{标题}

\subtitle{副标题}

% 答辩人
\author{你的名字}

% 指导教师
\teacher{老师名字\;职称}

% 答辩时间
\pubdate{\today}  %%%后期可以根据你的答辩时间进行更改

\begin{document}
	
	% 封面
	\maketitle
	
	% 目录页
	\begin{frame}{目录}
        \tableofcontents
	\end{frame}
	
	% 第一部分 转场
	\zzusection{第一章标题}{第一章副标题或者英文版标题}

    \begin{frame}[fragile, allowframebreaks]{算法命令使用说明}
        \begin{zzucode}[language=TeX]
\begin{zzubreakablealgorithm}
    \caption{算法名字} %算法的名字
    \label{算法标签}
    \begin{algorithmic}[1]
        \Require 输入
        \Ensure 输出
        \State 算法语句
        \For{i=1} For循环
            \State x=i+1
        \EndFor
        \If{a} 条件语句
            \State b
        \Else
            \State c
        \EndIf
        \State \Return x \Comment{返回语句}
    \end{algorithmic}
\end{zzubreakablealgorithm}
        \end{zzucode}
    \end{frame}

    \begin{frame}{命令说明}
        \begin{itemize}
            \item $\backslash$maketitle 显示封面;
            \item $\backslash$tableofcontents 显示目录;
            \item $\backslash$zzusection$\{\}\{\}$ 显示章节页;
            \item $\backslash$begin$\{$zzubreakablealgorithm$\}$ $\cdots$ $\backslash$end$\{$zzubreakablealgorithm$\}$ 显示跨页版本算法;
            \item $\backslash$begin$\{$zzucode$\}$ $\cdots$ $\backslash$end$\{$zzucode$\}$ 显示代码,且记得加上fragile命令;
            \item $\backslash$cite$\{\}$ 显示参考文献引用,例如\cite{barmpoutis2020review};
            \item $\backslash$thanksforlistening$\{\}$ 显示结束页。
        \end{itemize}
    \end{frame}
	
    \begin{frame}[allowframebreaks]{参考文献}
        \printbibliography
    \end{frame}

	% 谢谢收听页面
	\thanksforlistening{完结撒花!}
	
\end{document}

